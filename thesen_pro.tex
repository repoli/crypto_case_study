\section*{Thesen Pro}
\subsection{"'Terror Bekämpfung 1"'}
Ohne Überwachung ist ein Staat anfällig auf terroristische Angriffe und 
Attentate. Eine Überwachung der Kommunikationsmittel hilft, Angriffe 
frühzeitig zu erkennen und nötige Gegenmassnahmen einzuleiten. Es ist 
wichtig, dass international zusammengearbeitet wird und die 
Informationen untereinander ausgetauscht werden können. Es ist 
unverst\"andlich, weshalb viele Politiker diesen Austausch verhindern wollen. Die 
Verbrecher und Terroristen schlafen auch nicht: Falls der Staat die enge 
Zusammenarbeit nicht weiterentwicket, werden die Terroristen vernetzter sein als 
die Beh\"orden. Das ist das Gefährlichste, was uns passieren kann. Man sieht ja 
schon heute, wie eng gewisse mafiöse Strukturen zusammenarbeiten und 
global agieren. Sollten die Verbrecher den Kampf um die Informationen 
und Daten gewinnen, hat die Gesellschaft verloren. Dieser Entwicklung
muss mit allen Mitteln entgegengetreten werden.\\
Beispielsweise durch den Einsatz eines Key Escrow Verfahrens kann eine 
Strafverfolgungsbehörde bei zwingendem Verdacht jegliche 
verschlüsselten Daten entschlüsseln. Dies kann die Strafverfolgung in 
vielen Fällen effizienter gestalten. Der Besitz von Kinderpornographie 
würde ebenso aufgedeckt wie das Vorbereiten von terroristischen 
Anschlägen.\\
Die USA wird oftmals dafür kritisiert, dass die 
Einreisebestimmungen streng geregelt sind. Das stimmt, ist aber klar 
Teil unserer Strategie, alle nötigen Informationen zu sammeln, um einem 
terroristischen Verbrechen entgegenwirken zu können.

\subsection{"'Terror Bekämpfung 2"'}
Die verfügbaren Mittel, um Verbrechen 
aufklären zu können (oder noch besser, um sie verhindern zu können) 
müssen unbedingt ausgeschöpft werden. Die Überwachung von 
Kommunikationsmitteln, wie auch die Videoüberwachung im öffentlichen 
Raum, sind eine grosse Hilfe im Durchsetzen des geltenden Gesetzes. Es 
geht doch nicht an, dass beispielsweise in einer 
Bahnhofsunterführung Kameras installiert sind, aber die Bilder danach 
trotzdem nicht verwendet werden dürfen. Obwohl es ganz klar ist, das 
der Dieb während seiner Tat gefilmt wurde. Oder dass ein B\"urger zu Hause 
umgebracht wird und die Bilder des Nachbars, welcher den Täter 
zufälligerweise gefilmt hat, als Beweismaterial nicht anerkannt werden. 
Im schlimmmsten Fall wird dann noch der Nachbar bestraft, weil er ohne 
Befugnis gefilmt hat. Da entsteht der Eindruck, einige Leute wollten 
die Arbeit der Polizei massiv erschweren.\\
Insbesondere Kameras im öffentlichen Raum verbreiten
ein Gefühl der Sicherheit und viele Bösewichte werden von ihren 
Taten abgehalten. Vor allem die Jugendlichen sollten lernen, dass sie 
sich anpassen müssen und das sie im Auge behalten werden. Hier haben die 
Kameras eine gesunde, abschreckende Wirkung.\\
Man fragt sich, was die Menschen, welche sich gegen den Ausbau von 
Überwachung stellen, zu verbergen haben. Der Staat will ja nur Gutes 
für alle seine B\"urger. Wer nichts Illegales tut, hat auch nichts zu befürchten. 
Bürger, welche gegen die Überwachung sind, sind gegen das Gesetz. Im 
Endeffekt unterstützen sie Pädophile, Drogenhändler, Mafiosi - das 
ganze illegale Gesindel. Sie stellen sich gegen unsere Demokratie und 
gegen das Volk.

\subsection{"'Wir müssen mit der Zukunft gehen"'}
Google wird oftmals dafür kritisiert, dass sie Informationen sammeln. 
Das kann man ganz anders sehen. Google ist ein extrem zukunftsträchtiges, 
innovatives Unternehmen. In Zukunft wird nicht mehr Geld, sondern 
Information die Welt führen. Die Staaten sollten nicht gegen, sondern 
mit Google arbeiten. Sie würden von Googles Wissen profitieren können.
Es ist nicht zu verstehen, weshalb sie sich gegen eine Entwicklung stellen, welche 
nicht mehr aufzuhalten ist. Es geht nicht mehr darum, diese Bewegung 
stoppen, sondern sich ihr zu bedienen und darin der Beste zu sein. Es 
können nur die gewinnen, welche mit der Zukunft gehen.\\
Dies l\"asst sich an einem Beispiel verdeutlichen: Wir sind technisch in 
der Lage, Millionen von DNA-Datensätzen zu speichern, auszuwerten und 
miteinander zu vergleichen. Die Medizin könnte daraus wertvolle 
Informationen gewinnen, welche unglaubliche Fortschritte möglich machen 
würden. Es würden Lösungen gefunden werden, welche heute noch unvorstellbar 
sind. Die Medizin könnte innert kürzester Zeit riesige Schritte 
vorwärtsmachen, wenn diese Daten allen Forschern zur Verfügung gestellt 
würden. Aber leider wird dies noch immer bekämpft und von Politikern 
und bestimmten Organisationen verhindert. Die Menschen lassen sich von Bedenken 
hindern, dabei spielt hier die Zukunft. Je schneller die Menschheit sich anpasst, 
desto schneller profitiertn sie davon.\\
Wenn wir unser Wissen zusammentragen, können wir unser Leben viel 
angenehmer gestalten. Jeder nervt sich über Werbung, die er nicht haben 
will. Die Vorstellung, nur noch passende 
Werbung zu erhalten, gef\"allt. Das wäre doch vorteilhaft - man wird nur noch mit 
Informationen konfrontiert, die einen interessieren. Das wollen doch 
alle.\\
Die Auffassung von Privatsphäre wird sich stark 
verändern. Informationen, welche heute als privat und persönlich 
(also schützenswert) betrachtet werden, werden in Zukunft zum Wohle der 
Gesellschaft geteilt. So können alle vom Wissen der anderen profitieren.

\subsection{"'Sicherheit und Freiheit gehören zusammen"'}
Der Mensch kann sich nur dann frei fühlen, wenn er sich auch sicher 
fühlt. In der heutigen Gesellschaft drohen sehr viele Gefahren. 
Es ist die Aufgabe des Staates, seine Bürger zu (be-)schützen und ihnen 
das Gefühl von Sicherheit zu geben. Mit der massiven Zunahme von 
zugänglichen Informationen sinkt unser Sicherheitsgefühl. Von allen 
Seiten nehmen wir Bedrohungen war. Wir lesen über Verbrechen wie 
Diebstahl, sexuellem Missbrauch und Mord in unserer nächsten 
Umgebung. Wir sehen im Fernsehen die Kriege der ganzen Welt in Echtzeit 
und so weiter. Wir haben sehr viele Informationen, welche schwierig 
einzuschätzen sind. Diese Entwicklung werden wir nicht aufhalten 
können. Wir müssen uns ihr anpassen und so auch der Staat. Dieser muss 
Lösungen finden, wie unser Sicherheitsgefühl intakt bleiben kann und 
gestärkt wird. Eine Lösung ist der Einsatz von 
Kameras. Wenn diese überall präsent sind, fühlt man sich beschützt. Man 
ist dann nicht allein und weiss, dass ein Übeltäter beobachtet würde.\\
Ein anderes Beispiel ist das Überwachen von Datenverkehr im Internet. 
Einem Vater zweier Töchter ist wohler, wenn er weiss, dass in 
allen Chatrooms ein Polizisten mitlesen und im Notfall 
eingreifen.\\
Es ist extrem wichtig, dass Kinder in einer sicheren Umgebung 
aufwachsen. Wir alle tragen hier eine besondere Verantwortung. Die 
technischen Entwicklungen schreiten rasant voran, es stehen in immer 
kürzerer Zeit immer mehr Informationen zur Verfügung. Wir müssen 
versuchen, vorauszudenken. Es wäre fatal, wenn unsere Kinder alle diese 
Möglichkeiten nutzen könnten, aber die Sicherheit dabei verloren ginge.\\
Die Regierung ist gut beraten, wenn sie die Überwachung 
ausbaut. Ansonsten droht ihr ein Vertrauensverlust von Seiten der 
Bevölkerung.

\subsection{"'Ein Super-Chip für alle"'}
In Zukunft wird jeder von uns einen kleinen Chip auf 
sich tragen. Dieser würde unser Bewegungsprofil, unsere getätigten 
Einkäufe, medizinische Informationen und vieles mehr speichern. Ein 
solcher Chip würde Ärzten ganz neue Möglichkeiten bringen. 
Der Patient k\"onnte viel gerechter und individueller behandelt werden. Wenn 
einer mit Vergiftungserscheinungen eingeliefert würde, wüsste man, was 
er in den letzten Tagen eingekauft (und somit gegessen) hat. Man 
könnte viel effizienter reagieren. Wenn jemand verunfallt, wüsste man 
mit Hilfe des Chips, welche Blutgruppe die Person hat, welche 
Medikamente von ihr eingenommen werden, u.s.w. Wenn jemand im Ausland 
behandelt würde, könnten diese Informationen auch hier wieder 
ausgelesen und in die Behandlung miteinbezogen werden.\\
Bei Unfällen in unbewohntem Gebiet könnte anhand des Bewegungsprofils 
sofort herausgefunden werden, wo sich die Person genau befindet. 
Natürlich kann man dies heute beispielsweise bei Schneetouren-Fahrern 
auch. Aber es wäre sinnvoll, wenn alle Menschen davon profitieren 
würden und sie sich darüber gar keine Gedanken machen müssten.\\
Natürlich brächte ein solcher Chip noch viel mehr Vorteile mit sich: 
Wenn Eltern ihr Kind vermissen, besteht immer die Möglichkeit, es zu 
orten und es vor einem möglichen Verbrechen zu bewahren. Man denke hier 
zum Beispiel an eine Entführung. \\
Zudem könnte man auch Bankdaten auf dem Chip hinterlegen, damit man 
bargeldlos und ohne sich einen Pin merken zu müssen, bezahlen könnte. 
Man stelle sich vor, man hat diesen Chip im kleinen Finger implantiert 
und muss bei der Migros-Kasse nur kurz mit dem kleinen Finger über 
einen Scanner fahren. Das würde uns bestimmt auch das mühsame Anstehen 
an der Kasse ersparen.

