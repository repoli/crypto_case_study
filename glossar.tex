\section{Glossar}
	
	\subsection{Big Brother}
	  Imaginäre Person aus dem von George Orwell 1949 veröffentlichten 
	  Buch "1984". In diesem symbolisiert ''Big Brother'' den Überwachungsstaat und
	  dessen totalitäre Auswirkungen. 
	
	\subsection{Key Escrow}
	  Verschlüsselung ist die Umwandlung von Daten in eine Form, die von Drittpersonen nicht ohne 
	  Kenntnis einer weiteren Information, des sogenannten Schlüssels,
	  gelesen werden kann.
	  Key Escrow ist die vom Staat verordnete Speicherung sämtlicher zur Verschlüsselung 
	  benötigter Schlüssel. Dies hat zur Folge, dass 
	  Kryptographie ohne Hinterlegung der Schlüssel illegal sein muss. 
	
	\subsection{Privatsphäre}
	  Die Privatsphäre einer natürlichen Person ist ein im Gesetz 
	  festgeschriebenes Prinzip, welches als Menschenrecht betrachtet wird und 
	  juristisch unter gewissen Bedingungen eingeschränkt werden kann, z.B. 
	  im Strafrecht oder bei Personen öffentlichen Interesses. Es soll der 
	  Person Freiheiten geben, damit sie sich persönlich Entfalten 
	  kann, ohne Repressionen oder Einschränkungen fürchten zu müssen. 
	  Privatsphäre spielt heutzutage sowohl in der realen Welt ("physisch") 
	  als auch - und dies in zunehmendem Masse - im Internet eine Rolle 
	  ("virtuell"). 
	
	\subsection{Datenschutz}
	  Unter Datenschutz versteht man den Schutz personenbezogener Daten, d.h. 
	  Daten, die u.a. die Privatsphäre einer Person betreffen. Der 
	  Datenschutz ist in Europa auf Gesetzesebene geregelt und soll 
	  sicherstellen, dass jede einzelne Person über ihre eigenen Daten 
	  bestimmen kann. Dabei ist es irrelevant, ob die Daten sammelnde 
	  Institution der Staat oder eine Firma ist. 

	\subsection{Staatstrojaner}
	  Staatstrojaner im weiteren Sinn bezeichnet eine Art Software, welche auf 
	  dem Computer einer zu überwachenden Person installiert wird. Es geht 
	  dabei darum, verschlüsselte Daten und Datenströme vor der 
	  Verschlüsselung oder nach der Entschlüsselung zu lesen. Eingesetzt 
	  wird der Staatstrojaner von Strafverfolgungsbehörden. 
	
	\subsection{Überwachung}
	  \subsubsection{Telekommunikationsüberwachung (TKÜ)}
	    Telekommunikationsüberwachung (Englisch: Lawful Interception) 
	    bezeichnet die vom Staat durch die Legislative befohlene Möglichkeit, 
	    Ermittlungs- und anderen Behörden Zugriff auf Verkehrsdaten und/oder 
	    Dateninhalte von Telekommunikationsinfrastruktur zu ermöglichen. 
	    Verkehrsdaten (auch Verbindungsdaten) beinhalten z.B. Zeitpunkt, Ort, 
	    Teilnehmer und Dauer eines Telefongesprächs, nicht aber den 
	    Dateninhalt, also die gesprochenen Worte. Die Analyse von Verkehrsdaten 
	    wird auch Verkehrsflussanalyse (Englisch: Traffic Analysis) genannt. 
	
	  \subsubsection{Echtzeitüberwachung }
	    Damit ist der unmittelbare Zugriff auf durch Überwachung gewonnene 
	    Daten gemeint. Z.B. können Ermittler bei einem Telefongespräch 
	    gleichzeitig mit den Teilnehmern mithören. 
	  
	  \subsubsection{Videoüberwachung}
	    Das kontinuierliche Filmen von Orten durch Videokameras und oft 
	    Speicherung der aufgezeichneten Daten. Die Videoüberwachung wird zur 
	    Verbrechensprävention und nachträglichen Ermittlung von Tätern 
	    eingesetzt. So die Theorie. 
	  
	\subsection{Vorratsdatenspeicherung}
	  Das Erheben und Speichern von Daten ohne Verdacht, für die spätere 
	  Verwendung von Strafverfolgungs- und anderen Behörden. Sie wird oft im 
	  Gesetz mit Mindest- oder Maximalspeicherdauer festgelegt. 
	
	\subsection{Cyberkrieg}
	  Ein Begriff, dessen Bedeutung noch nicht eindeutig festgelegt worden 
	  ist. Wird oft als Erweiterung des konventionellen Krieges auf 
	  Informationsnetzwerke und dessen virtuellen Räume gesehen. Die 
	  möglichen Vorgehensweisen reichen von Propaganda bis zur Sabotage, 
	  welche durchaus in der realen Welt Auswirkungen haben kann. 
