\section{99 Gramm Überwachung. Darf es noch etwas mehr sein?}
% Schweizerdeutsch?
"`Wieviel hätten Sie denn gerne?"', wird man Sie schon
häufiger an der Käsetheke gefragt haben. Mit Vorfreude haben
Sie noch einen Blauschimmelkäse dazu genommen. Mehr ist ja
schliesslich besser. Oder nicht?\\
Wieviel Überwachung sollte ein Staat vornehmen? Wann ist
der \textit{Big Brother} eine Hilfe, wann wird er zur Gefahr?
Handelt es sich überhaupt um einen Überwachungs\textbf{\textit{staat}}
oder ist Orwells Vision der Privatisierung zum Opfer gefallen?\\
Am 30.11.2011 findet eine Podiumsdiskussion auf höchstem Niveau statt:
Bekannte Vertreter aus Politik, Wirtschaft
und Wissenschaft finden sich zusammen, um über diese Fragen zu diskutieren.\\ 
Der Abend wird von dem prominenten Moderator Anton R. Müller
geleitet. Herr Müller, renommierter Sprachwissenschafter und Politologe
an der Universität Basel, hat uns in
einem Interview bereits einen kleinen Einblick auf die
bevorstehende Podiumsdiskussion gegeben:

\begin{quote}
\textbf{NZZ:} Herr Müller, die angekündigte Podiumsdiskussion zum Thema
"`Der Staat als Big Brother?"' hat bereits im Vorfeld zu
sehr unterschiedlichen Reaktionen geführt. Thomas Würgler, 
Kommandant der Kantonspolizei Zürich kann sich beispielsweise "`[...] nicht
 vorstellen, dass jemand an der Notwendigkeit und am Nutzen
 einer umfassenden Überwachung zum Wohle der Bürger zweifelt. [...]"'
Wie kam es zu dieser Reaktion und warum braucht es eine  
Podiumsdiskussion zu diesem Thema?
\end{quote}
\begin{quote}
\textbf{Müller:} Zunächst einmal möchte ich ankündigen, dass
der von Ihnen zitierte Herr Würgler, nicht zuletzt aufgrund
seiner Erfahrungen aus dem Berufsumfeld, einer unserer Top-Refe\-renten
an diesem Abend sein wird. Der von Ihnen zitierte Abschnitt stammt
bekanntlich aus der \textbf{TALKTÄGLICH}-Sendung vom September,
in der er mit Hanspeter Thür,
dem eidgenössischen Datenschutzbeauftragten, über die zukünftige
Entwicklung der Videoüberwachung auf öffentlichen Toiletten
diskutiert.
Der Lehrstuhl für Gesellschaftswissenschaften der Universität Basel
hat diese Entwicklung, welche bei vielen Menschen gemischte Gefühle
hervorruft, zum Anlass genommen, eine Podiumsdiskussion zu veranstalten.
Dabei sollen Vor- und Nachteile, Gefahren und Nutzen, sowie der aktuelle
 Stand und die Entwicklung der Überwachung zur Sprache gebracht werden.
\end{quote}
\begin{quote}
\textbf{NZZ:} Besten Dank für die kurze Übersicht,
Herr Müller.
\end{quote}
\begin{quote}
\textbf{Müller:} Gern geschehen.
\end{quote}
In diesem Sinne überlassen wir Ihnen, liebe Leser und Leserinnen,
die Frage "`Wieviel hätten Sie denn gerne?"' und freuen uns, Sie am
30.11.2011 an der Universität Basel zur Podiumsdiskussion
begrüssen zu können.
